%%%%%%%%%%%%%%%%%%%%%%%%%%%%%%%%%%%%%%%%%%%%%%%%%%%%%%%%%%%%
%%% CHECKLIST TEMPLATE FOR NEGATIVE DATA INITIATIVE
%%% ADAPTED FROM LIVECOMS ARTICLE TEMPLATE (3/19/2023)
%%%%%%%%%%%%%%%%%%%%%%%%%%%%%%%%%%%%%%%%%%%%%%%%%%%%%%%%%%%%
%%% PREAMBLE
\documentclass[9pt,lessons]{livecoms}
% Use the 'onehalfspacing' option for 1.5 line spacing
% Use the 'doublespacing' option for 2.0 line spacing
% Use the 'lineno' option for adding line numbers.
% Use the 'pubversion' option for adding the citation and publication information to the document footer, when the DOI is assigned and the article is added to a live issue.
% The 'bestpractices' option for indicates that this is a best practices guide.
% Omit the bestpractices option to remove the marking as a LiveCoMS paper.
% Please note that these options may affect formatting.

\usepackage{lipsum} % Required to insert dummy text
\usepackage[version=4]{mhchem}
\usepackage{siunitx}
\DeclareSIUnit\Molar{M}
\usepackage[italic]{mathastext}
\graphicspath{{figures/}}
\usepackage{amsthm}
\theoremstyle{definition}
\newtheorem{definition}{Definition}[section]

\theoremstyle{remark}
\newtheorem*{remark}{Remark}

%%%%%%%%%%%%%%%%%%%%%%%%%%%%%%%%%%%%%%%%%%%%%%%%%%%%%%%%%%%%
%%% IMPORTANT USER CONFIGURATION
%%%%%%%%%%%%%%%%%%%%%%%%%%%%%%%%%%%%%%%%%%%%%%%%%%%%%%%%%%%%

\newcommand{\versionnumber}{0.1}  % you should update the minor version number in preprints and major version number of submissions.
% Do not add a newline in the next command, no matter how long the repository name is, as it will break the link in the PDF.
\newcommand{\githubrepository}{\url{https://github.com/Poruthoor/NegativeDataInitiative}}  %this should be the main github repository for the article.

%%%%%%%%%%%%%%%%%%%%%%%%%%%%%%%%%%%%%%%%%%%%%%%%%%%%%%%%%%%%
%%% ARTICLE SETUP
%%%%%%%%%%%%%%%%%%%%%%%%%%%%%%%%%%%%%%%%%%%%%%%%%%%%%%%%%%%%
\title{Negative Data Initiative [Letter v\versionnumber]}

\author[1*]{Trainee collective}

\blurb{This checklist document is maintained online on GitHub at \githubrepository; to provide feedback, suggestions, or help improve it, please visit the GitHub repository and participate via the issue tracker.}

%%%%%%%%%%%%%%%%%%%%%%%%%%%%%%%%%%%%%%%%%%%%%%%%%%%%%%%%%%%%
%%% ARTICLE START
%%%%%%%%%%%%%%%%%%%%%%%%%%%%%%%%%%%%%%%%%%%%%%%%%%%%%%%%%%%%

\begin{document}

\begin{frontmatter}

\maketitle

\begin{abstract}
   An ongoing community letter for Negative Data Initiative. 

   Current focus : Negative Data in Computational Biophysics. 
\end{abstract}

\end{frontmatter}


% This provides a checklist which
% - spans a full page
% - consists of multiple sub-checklists
% - exists on a separate page
% This style of checklist will be especially helpful if you want to encourage readers to print and use your checklist in practice, as they
% can easily print it without also printing other material from your manuscript. However, other styles of checklist are also possible (below).
\begin{Checklists*}[p!]

\begin{checklist}{Initial hypothesis}
\begin{itemize}
\item An initial testable hypothesis backed up by strong rationale is formed ( based on literature review, prior knowledge, research gap identified, etc.)
\item A detailed, careful experiment design is formed to test the hypothesis with statistical rigor.
(Included independent replicates, following best practices in the community, ...) 
\item Benchmarks for testing the validity of results is set up a priori.
\end{itemize}
\end{checklist}

\begin{checklist}{Analysing the data}
\begin{itemize}
\item Sanity checks have been done to the best of your abilities.
Code/method review, debugging, testing..etc. Independent replicates   
\item Identified or tried different analysis schemes when the initial scheme failed the anticipated results.
\item Additional testing scheme was added to check for a potential flaw in the design
\item Considered alternative hypotheses and tested them
\end{itemize}
\end{checklist}

\begin{checklist}{Interpreting the data}
\begin{itemize}
\item You have not done any ``cherry-picking" of data. Analysis schemes are done on data without any bias
\item Your analysis indicates the rejection of your initial hypothesis with statistical rigor.
Or, you see an absence of anticipated features or characteristics that helped you formulate your hypothesis.
Including simple, semiquantitative checks that describe them.
\end{itemize}
\end{checklist}

\begin{checklist}{Reporting the data}
\begin{itemize}
\item Honest and complete reporting of all the attempted tests and rationale
\item Report alternative hypothesis tested if any. Discuss such possibilities.
\item Discuss the potential flaws, pitfalls, and other features of the method that was used.
\end{itemize}
\end{checklist}

\end{Checklists*}

\section{Why this initiative?}

The core idea is to help the trainees to showcase the negative results and pitfalls they encountered while
having good visibility and feedback from the community.
Based on an \href{https://twitter.com/poruthoor/status/1635740205373087744?s=20}{earlier tweet discussion},
a group of trainees came towards to brainstorm on this idea.
The group thinks that the following model might work better:\\

\textbf{Short-term goals:}

1. Form a community consensus on the definition of negative data

2. Once the consensus is made, prepare a checklist for the trainees to follow to make sure what they have is negative data

3. Form an advisory board to assist the trainee group by inviting senior members of the community

4. Initial call for moderators/volunteers to the trainee community to organize a virtual webinar/conference.
This virtual gathering allows the trainees to discuss and share their negative data.
Applications to the virtual gathering are based on abstract submission.
This call for volunteers is to organize the event.
The advisory board will be kept in the loop during this entire process\\

\textbf{Long term goals:}

5. Organizing the event

6. Potential publication via LiveCoMs ``lessons learned" article by grouping the negative data

7. Use this model to advocate for more visibility in major conferences\\

The group decided to run a pilot program of this model with the computational biophysics community since the initial core trainees and potential advisory board members will be coming out of this community.



\section{Discussion}

The trainee group is requesting feedback from the community on (1) the overall model,
(2) the definition of negative data, and (c) a potential (LiveCoMs-like) checklist for trainees to
identify the negative data at their hands.
Following the LiveCoMs protocol, we will use the Github issue tracker for feedback, suggestions, and further discussions.

\subsection {Feedback on the overall model}
Regarding the feedback on our proposed model, described in the previous section, the discussions are hosted here at \href{https://github.com/Poruthoor/NegativeDataInitiative/issues/2}{github issue \#2}

\subsection {Feedback on negative data definition}
The trainee group discussed on the definition of negative data and suggested the following.
Again, we are using this to initiate a discussion and will finalize it after extensive feedback from the community.

\begin{definition}
Negative Data does not support the initial hypothesis or expected outcome after careful and reproducible meticulous scientific research.
However, the authors believe that this can drive science forward or are lessons learned that help future trainees while designing their experiments.  
\end{definition}

We hope to reach a clear, concise definition of negative data through the discussions hosted here at \href{https://github.com/Poruthoor/NegativeDataInitiative/issues/3}{github issue \#3}.
We ask the community to share their definition of negative data here.

\subsection{Feedback on the checklist}

From the brief discussion during the brainstorming meeting, we propose a general outline for a checklist to identify negative data and invite community suggestions.
Please find the checklist on the next page.
The discussion for the checklist feedback is hosted here at \href{https://github.com/Poruthoor/NegativeDataInitiative/issues/4}{github issue \#4}

\end{document}
